\chapter{Outlook}

\section*{More of current features}
The easiest step to add more value to the current application is by extending existing features.

\subsection*{New nutritional goals}
New goals can broaden the target audience. One could open up to the medical market by introducing a strict reduction of sugar for diabetics or lowering salt and cholesterol for people suffering from heart diseases. The implementation would be fairly simply. The algorithm could filter possible substitutes by their content of salt instead of their NutriScore.

\subsection*{Add ingredients and recipes}
Foodo should know about as much ingredients as possible, to consider them in the substitution algorithm. This directly improves the proposed substitutes because there are more alternatives to choose from. New recipes on the other hand are essential to attract and keep users. We claimed to provide users with their favorite recipes to spare them from creating them manually. By delivering on this promise, we can embrace Foodos convenience.

\section*{Community}
The task of adding recipes and ingredients to the database could be transferred to the user.
With a rating system, popular recipes could rise to the top and bad ones would be sorted out. This system could also be applied to the substitutes where users like or dislike existing ones and can propose new ones. The data which substitute a user picks and if the user liked it or not is already collected and just has to be aggregated and turned into a popularity ranking.
With features like comment sections, users could exchange more tips regarding certain substitutes (how good is it, how to prepare it, how much of the original ingredient should be replaced with it). That way, a first collection of substitutes could emerge, supporting further research in this field. Moreover, the focus on the community would get users more involved and keep them attached to foodo.

\section*{Recommendations}
Once there is a stable user base, regularly interacting with foodo, the arising data can be used for analytical purposes. Besides optimizing substitution proposals, new recipes could be recommended to users, based on what they have already tried and liked. Adapting to a users preferences would lead to a more personalized experience.

\section*{Complements}
An easier approach to recipe improvement is to use complementary ingredients instead of substitutes. Complements only have to fit well to the existing dish, but don't have to replace an ingredient in multiple regards (binding/loosening properties, moisture, taste, color). 
For example peppers fit well to a quiche, stir-fried vegetables or ratatouille but are in no regard a substitute for one of the ingredients in these dishes. The same applies to adding a handful of peas to pork with noodles. An addition that can’t be considered a replacement but is still an improvement, at least indirectly. Because the peas increase the total weight and therefore the relative composition of the dish. This can be directly seen in the nutritional information of a single serving (example/ figure below). 

\begin{table}[H]
	\hspace{-5pt}
	\begin{scriptsize}
		\begin{tabularx}{\textwidth + 5pt}{| @{\hspace{3pt}} M | @{\hspace{3pt}} M | @{\hspace{3pt}} M | @{\hspace{3pt}} M |}
		\hline
		\textbf{Nutrient} & \textbf{Pork} & \textbf{Noodles} & \textbf{Peas}\\
		\hline
		Calories & 271 kcal & 137 kcal  & 81 kcal\\
		\hline
		Fat & 17g & 2g  & 0,5g\\
		\hline
		Carbs & 0g & 25g  & 14,5g\\
		\hline
		Protein & 27g & 4,5g  & 5,5g\\
		\hline
	\end{tabularx}
	\end{scriptsize}
	\caption{Ingredients and their nutritional values}
	\vspace{1em}
\end{table}

\begin{table}[H]
	\hspace{-5pt}
	\begin{scriptsize}
		\begin{tabularx}{\textwidth + 5pt}{| @{\hspace{3pt}} M | @{\hspace{3pt}} M | @{\hspace{3pt}} M | @{\hspace{3pt}} M |}
		\hline
		\textbf{Nutrients} & \textbf{Without Peas} & \textbf{With Peas} & \textbf{Difference}\\
		\hline
		Calories & 754 kcal & 674 kcal  & 81 kcal\\
		\hline
		Fat & 31g & 25g  & 6g\\
		\hline
		Carbs & 62g & 61g  & 1g\\
		\hline
		Protein & 53g & 47g  & 6g\\
		\hline
	\end{tabularx}
	\end{scriptsize}
	\caption{Improvement of one serving (400g)}
	\vspace{1em}
\end{table}

Of course this doesn’t work out if still the same amount of pork and pasta is eaten and the peas just come on top. But in theory one would reduce either the pork, the noodles or both, to maintain the same size for a single serving.

\section*{Take less}
The aforementioned idea of complements builds upon reducing the amount of (unhealthy) ingredients using supplements instead. But there are also use cases in which neither substitutes nor supplements are needed to improve a recipe. Especially in the context of baking there are often huge amounts of sugar used, which can be reduced without having any noticeable impact on the pastries. The dispensable amount of the unhealthy ingredient could either be provided and verified by the community or automatically inferred by comparing other recipes of the same dish.

\section*{Business model}
In the future it’s also worth thinking of new income streams, besides the current premium membership on a voluntary basis. One source could be advertisements, like banner ads or product placements. Kitchen gadgets or ingredients by a certain brand could be mentioned within recipes or their preparation steps. 
Partnerships with food retailers would provide a stable income and additional input for the ingredient database. Edeka, Rewe and their competitors are already storing nutritional facts, geographical information (where the product comes from) as well as the current price and could grant us access to them. With these information optimizing a recipe in terms of overall costs or eco-friendliness (prefer regional or seasonal products) could become a new feature. 
Further, this would make it easier to implement a shopping list, directly linked to a partners delivery service.