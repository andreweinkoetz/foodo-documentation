\chapter{Evaluation}
We conducted two big evaluation rounds. The first one targeted the first prototype and the second one the final prototype. The interviews were conducted either in a personal talk or via a messenger. Due to our setup, in which the most current version of Foodo is deployed on Amazon Web Services (AWS) and accessible for the public, we just handed the link to our webapp to the participants and they could explore Foodo on their own with their own device. That way, we could also test how good Foodo perform on different devices with different browser which strengthens our cross-platform and multi-device approach. The two big evaluation rounds each took up to two to three weeks. This enabled us to incorporate the first gathered feedbacks and to test the new changes within the same evaluation round. This was only feasible because we developed Foodo with a focus on agile development and relied on continuous integration and continous delivery so that even small change get delivered and deployed to the production system as soon as possible. 

Overall, nine (\textbf{ZAHL NICHT FINAL}) participants (seven in the first round and two (\textbf{ZAHL NICHT FINAL}) in the second round) tested Foodo and provided us feedback. They liked Foodo a lot and most of them would use Foodo, when it is market-ready. Specifically, they praised Foodo's clean design, its corss-platform multi-device approach, the Alexa integration, and the idea of getting recommendations for substitutions which improve the healthiness of recipes. Their feedback or suggestions for improvement can be categorized in three areas, namely functional, visual and error feedback, and it is condensed into the following paragraphs.

\section{Functional Feedback}

\smallheadline{Implemented Feedback}

substitution decision is final -> editing needed

difficulty of recipes

Recipe categories with nice overview page -> Kachel

explainatory page about what Foodo is and how it works

list of favorite recipes

statistics on achievements through using Foodo

selecting specific dislikes (some people eat all kinds of meat except porc meat)

\smallheadline{Not Yet Implemented Feedback} 

adding more allergies (nuts) 

adding more lifestyles (kito)

more recipe categories (fast, cheap, feast)

rate recipes with stars

extension with community functions like commenting on recipes and substitutes or posting what you have eaten or substituted

portion calculator for nutrition values and amounts of ingredients

insert own recipes (only implemented in the admin panel)

recommend recipes to user based on the ones he liked / based on clocktime

recommend fitting desert and starter

say reason for substitution (alllergy, dislike, unhealthy, ...)

offer challenges for nudging

offer grocery list

save recipes

suggest crazy substitutions which you wouldnt expect to work e.g. from normal latte to  Red Beet Latte

offer timer for support during cooking

Kosten ausrechnen, wie viel es ungefähr kostet für z.B. 2 Personen???
    
\section{Visual Feedback}

\smallheadline{Implemented Feedback}

The visual feedback which we already implemented mainly consists of four remarks. The first one is that not every part in the intermediate prototypes was internationalized. Secondly, at the beginning Foodo did not use standardized abbreviations for ingredient amounts, like "ml" and "g". Thirdly, Foodo should put more visual emphasis on the NutriScore. This is achieved by displaying the NutriScore of every substitute and by coloring them appropriately. Fourthly, the diagrams were praised for nicely conveying the important nutritional information. 

\smallheadline{Not Yet Implemented Feedback} 

The participants noted several times that Foodo should use common units, like a pinch of salt instead of 0.02g. This would improve the usability a lot. Another important remark is that one of the diagrams cannot convey its message if the user suffers a red-green color blindness because the diagram is colored in red and green. 

\section{Error Feedback}

\smallheadline{Implemented Feedback}

The participants reported that some errors regarding the results of the substitution algorithm, e.g. they selected "Lactose" as allergy and the the algorithm still recommended them to use yogurt as a substitute. This was due to the fact that in the first tests the algorithm was mocked by using randomized categories. The other recognized errors were that it was not possible to select "None" in allergies and some alignment errors on some of the user's devices, like that a button was moved outside the screen.

\smallheadline{Not Yet Implemented Feedback} 

In our current version, some of the participants experienced that on their  device with their browser that there sometimes still some displaying errors. One participant experienced that in his setup the keyboard overlays the input field so that inserting the required input can be quite cumbersome. Another participant discovered in his setup that the alignments in the nutrition table gets destroyed by one word seems to be too long. Apart from that they did not discover any other errors.


