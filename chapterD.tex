\chapter{Evaluation}
We conducted two big evaluation rounds. The first one targeted the first prototype and the second one the final prototype. The interviews were conducted either in a personal talk or via a messenger. Due to our setup, in which the most current version of Foodo is deployed on Amazon Web Services (AWS) and accessible for the public, we just handed the link to our webapp to the participants and they could explore Foodo on their own with their own device. That way, we could also test how good Foodo performs on different devices with different browsers which strengthens our cross-platform and multi-device approach. The two big evaluation rounds each took up two to three weeks. This enabled us to incorporate the first gathered feedbacks and to test the new changes within the same evaluation round. This was only feasible because we developed Foodo with a focus on agile development and relied on continuous integration and continuous delivery so that even small change could get delivered and deployed to the production system as soon as possible. 

Overall, nine (\textbf{ZAHL NICHT FINAL}) participants (seven in the first round and two (\textbf{ZAHL NICHT FINAL}) in the second round) tested Foodo and provided us feedback. They liked Foodo a lot and most of them would use Foodo when it is market-ready. Specifically, they praised Foodo's clean design, its cross-platform multi-device approach, the Alexa integration, and the idea of getting recommendations for substitutions which improve the healthiness of recipes. Their feedback or suggestions for improvement can be categorized in three areas, namely functional, visual and error feedback, and it is condensed into the following paragraphs.

\section{Functional Feedback}

\smallheadline{Implemented Feedback}

Regarding the functional feedback which we already implemented, the participants' feedback can be summarized into seven main points. 
Firstly, they mentioned that the substitution decision is final and suggested that the recipe and the corresponding substitutions should be editable. 
Secondly, the participants highlighted that it is common to display the difficulty of a recipes and therefore they would expect that of an app offering recipes. 
Thirdly, they would love to see the recipes in a box grid overview with nice pictures so that they can scroll through when searching for inspiration on what to cook. 
Fourthly, some of the participants did not understand what Foodo exactly does and therefore suggested that there should be an explanatory page on what Foodo is and how it works. 
Fifthly, they asked for a list of the dishes they cooked last. 
Sixthly, the participants mentioned that it would be motivating if they could see statistics on how there nutrition improved through the usage of Foodoo.
Lastly, they praised the possibility to select dislikes on the ingredient level, because some of the eat all kinds of meat except porc meat. With our dislikes functionality, we can incorporate this specific taste.

\smallheadline{Not Yet Implemented Feedback} 

The participants suggested a whole bunch of new features and wished extensions. These can be categorized in extension of current features, community functionalities, offering recommendations and new other features. The suggestions for extension of current feature focuses of offering "more", meaning more allergies, e.g. nuts, more lifestyles, like "Paleo", more recipe categories, like "fast", "cheap", or "holiday", and more recipes. In regard of community functionalities the participants wished to be able to rate recipes and substitutions with stars, to comment and to see comments on substitutions and to post what they have eaten or substituted so that other can see it. Another wished feature are the services offered by a recommender system. The participants would like get recommendations for recipes based on the ones they like or based on the time of the day. They also would love to get informed about recipes similar to the one they are looking at and about starters and deserts fitting to the current recipe if it is a main dish. During the feedback  sessions, the participants suggested even more potential new features. Among these are that they would like to use a portion calculator which also updates the fact table and the diagrams accordingly, to insert own recipes (which we already implemented but just for the admin panel), to get a automatically generated grocery list and to bookmark recipes. Additionally, they suggested that Foodo could offer challenges in which they can participants, offer an integrated timer for supporting the cooking activity, and that Foodo should explain the reason why a certain subsititution is offered so that they can also educate themselves on healthy nutrition by cooking with Foodo. Furthermore, one participant mentioned that Foodo should also try to experiment with substitutions which seem to be a bit crazy at first glance. An example is that starting from a Cafe Latte Foodo suggests through substitution a Red Beet Latte which is actually a completely different drink.
    
\section{Visual Feedback}

\smallheadline{Implemented Feedback}

The visual feedback which we already implemented mainly consists of four remarks. The first one is that not every part in the intermediate prototypes was internationalized. Secondly, at the beginning Foodo did not use standardized abbreviations for ingredient amounts, like "ml" and "g". Thirdly, Foodo should put more visual emphasis on the NutriScore. This is now achieved by displaying the NutriScore of every substitute and by coloring them appropriately. Fourthly, the diagrams were praised for nicely conveying the important nutritional information. 

\smallheadline{Not Yet Implemented Feedback} 

The participants noted several times that Foodo should use common units, like a pinch of salt instead of 0.02 g. This would improve the usability a lot. Another important remark is that one of the diagrams cannot convey its message if the user suffers a red-green color blindness because the diagram is colored in red and green. 

\section{Error Feedback}

\smallheadline{Implemented Feedback}

The participants reported that some errors regarding the results of the substitution algorithm, e.g. they selected "Lactose" as allergy and the the algorithm still recommended them to use yogurt as a substitute. This was due to the fact that in the first tests the algorithm was mocked by using randomized categories. The other recognized errors were that it was not possible to select "None" in allergies and some alignment errors on some of the user's devices, like that a button was moved outside the screen.

\smallheadline{Not Yet Implemented Feedback} 

In our current version, some of the participants experienced that on their  device with their browser that there sometimes still some displaying errors. One participant experienced that in his setup the keyboard overlays the input field so that inserting the required input can be quite cumbersome. Another participant discovered in its setup that the alignments in the nutrition table gets destroyed by one word seems to be too long. Apart from that they did not discover any other errors.


