\chapter{Evaluation}
We conducted two big evaluation rounds. The first one targeted the first prototype and the second one the final prototype. The interviews were conducted either in a personal talk or via a messenger. Due to our setup, in which the most current version of Foodo is deployed on Amazon Web Services (AWS) and accessible for the public, we just handed the link to our webapp to the participants and they could explore Foodo on their own with their own device. That way, we could also test how good Foodo perform on different devices with different browser which strengthens our cross-platform and multi-device approach. The two big evaluation rounds each took up to two to three weeks. This enabled us to incorporate the first gathered feedbacks and to test the new changes within the same evaluation round. This was only feasible because we developed Foodo with a focus on agile development and relied on continuous integration and continous delivery so that even small change get delivered and deployed to the production system as soon as possible. 

Overall, nine (\textbf{ZAHL NICHT FINAL}) participants (seven in the first round and two (\textbf{ZAHL NICHT FINAL}) in the second round) tested Foodo and provided us feedback. This feedback can be categorized in three areas, namely functional, visual and error feedback, and it is condensed into the following paragraphs.

\section{Functional Feedback}

Man kann nicht zurücksubstituieren (die Entscheidung ist endgültig)

Schwierigkeitsgrad bei Rezepten 

    Gute Aufmachung ansonsten, wenn auch einige Features vermisst werden (just for the look/ sieht alles noch sehr plain aus - evtl. Kochhistorie, Merkzettel, Einkaufszettel, Zutatensuche..)
    
    Kategorien(dessert, vorspeise und  co): Ja das sehe ich jeweils über den Rezepten, genau. Nur hätte ich an eine extra "Kategorie"/Page (sorry, noob) oben neben dem Home/Profil gedacht. Wenn man wirklich nicht weiß was man will und mal inspiration braucht 
    
        Nutriscore Erklärung als intro
        
    Nüsse fehlen bei Allergenen
    
        Super sind die Nährwertangaben - ganze Fleißarbeit wäre es, wenn ihr noch so nen Portionenrechner hättet der dann auch gleich die nährwerte mitberechnet und die Zutatenmenge anpasst!
        
        Bewertung der Rezepte (1-5 Sterne)
        
    Schwierigkeitsgrad der Rezepte
    
    Kommentarfunktion bzw direkte Feedback Funktion bezüglich der Substitute und Verbesserungsvorschläge
    
    Rezepte selber einreichen
    
    Portionen ausrechnen
    
    Nährwerte entsprechend der Portionen
    
    Mehr Kategorien für Rezepte: schnell, kostengünstig, Feiertagsessen
    
    Ähnliche Rezepte anbieten
    
    Was passt dazu? Vor und Nachspeise dazu
    
    Speichern von Rezepten
    
    Einkaufsliste anbieten vor allem für Alexa
    
    Gimmik: Substitut anbieten, dass sehr weit weg ist, damit komplett neues Gericht entsteht (Red Beet Latte)
    
    Favoritenliste von Rezepten
    
    Direktes Feedback 
    
    Selber Rezepte einstellen
    
    Community. Posten was man substituiert hat -> hat richtig geil geschmeckt
    Kommentare
    
    Info über nutriscore
    
    Sterne Rating 

Diese Woche Statistik über eingesparte Sachen

    Challenges -> nudging
    
    Grund für Substitution angeben (Allergie, Dislike, Unhealthy)
    Info über nutriscore
    
     Substitution wieder entfernen
     
      Rezept Vorschläge basierend auf Uhrzeit 
      
      Idee ist gut
      
      mehr lifestyles (kito)
      
genaues Auswählen von Dislikes ist gut (-> nicht Vegetarian, aber isst kkein Schweinefleisch)

    Finde alles super, macht jetzt nen deutlich stimmigeren Eindruck! 
    
    Auch die Diagramme finde ich top
    
        Gute Arbeit  :)

\section{Visual Feedback}

Internationalization ist noch nicht komplett (e.g. Goals)

Einheiten der Rezepte teilweise inkonsistent angegeben (ml ist abgekürzt, gramm ausgeschrieben)

Prise Salz besser als 0,02 gr
    Normgröße: 400gr Tomaten -> 2,5 Tomaten

Laden von Bildern dauert erstaunlich lang

    Laden von subsitutes dauert bissi, find ich generell geil aber könnte man locker auch mal für alle Fleischsachen/tierische Produkte ergänzen! 
    
        Timer Funktionalität anbieten. Kleine Schaltfläche die einen Timer aktiviert. App Wechsel ist ziemlich nervig
        
        Kosten ausrechnen, wie viel es ungefähr kostet für z.B. 2 Personen
        
        Nutriscore mehr hervorheben -> durch Transparenz des Bilds
        
        Aussehen toll
        
        einheiten abkürzen -> sonst zu lang
internationalization ist bei nährwerten noch nicht da

NutriScore erhält zu wenig Aufmerksamkeit

Farbwahl berücksichtigt nicht rot-grün schwäche

    Finde alles super, macht jetzt nen deutlich stimmigeren Eindruck! 
    
    Auch die Diagramme finde ich top

\section{Error Feedback}

Registrierung verlangt beim Username min 2 Zeichen und akzeptiert aber nicht 2 Zeichen

Substitute Buttons rutschen beim Handy raus

Bei Allergien kann man nicht none wählen

Rezept bearbeitet mit Application macht keinen Sinn

    Also wenn man Vegan auswählt und wieder zum Suchfeld geht, werden einem da lauter Fleisch/Eierrezepte vorgeschlagen. 
    
     Zurück von about us zu Home landet man nicht beim vorher angeschauten Rezept, sondern am Anfang

 Pw ändern ausloggen einloggen -> man landet auf der pw änder Seite 

bei manchen handys ist die Tastatur beim Eintippen im Weg bei den Gerichten

 Einziges Manko: die Tabelle wird durch ein Wort verzogen (s. Screenshot). 



