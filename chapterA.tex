\chapter{Introduction}
\section{Motivation}
Awareness and interest in healthy nutrition is on the rise. The demand for organic and regional food is larger than ever. Ethical reasons like climate change or animal welfare aren’t the only driving factors of this development. And this development isn’t restricted to athletes and wealthy households either. There has never been a larger variety of dietary plans and advice. 

Technical innovations helped to shorten the time spent in the kitchen and enabled passionate amateur chefs to prepare dishes that used to be time consuming, in a convenient and fast way. The availability of ingredients increased immensely. Due to globalization and the industrialization of agriculture, the majority of food items became affordable for most people. The product range for vegans, vegetarians and eaters suffering from allergies and intolerances has also broadened. 

With all these positive developments, one could come to the conclusion that it has never been easier to have a healthy, personalized diet. Unfortunately, for some reason not everyone seems to use this opportunity and sticks to old habits instead.
\section{Problem statement}
When we were collecting opinions and practices regarding eating habits in our circle of friends and acquaintances, as well as online forums, we often came across similar statements. 

A surprising majority of the respondents were quite aware of their eating behavior. Moreover, most of them had a rough understanding what constitutes an unhealthy diet and were already trying to avoid fast food. Therefore, we decided to target this particular group of people that is motivated to improve their nutrition. 

By cooking at home, instead of eating out or ordering food, they already took the first step towards a healthier lifestyle but even then it’s important what ends up on the plate. At that point, there is often an obstacle, hindering those people to further improve their diet, leading us to their pains.

\subsection*{Pains}
According to a study of the Techniker Krankenkasse from 2017 \autocite{katja_wohlers_iss_2017}, only 2\% of respondents don't have interest at all in healthy nutrition. But on the other hand, only 10\% claim to eat healthy enough. 

The rest is stuck in the middle, seeing themselves faced with obstacles. This can be the lack of time (56\%), motivation (~45\%), money (~30\%), or knowledge (25\%). The latter is understandable, since the wide variety of contradicting opinions in the complex subject of nutrition can have a discouraging effect. Gathering new recipes is also quite time consuming. Without enough cooking skills this won't feel rewarding either, because trying out something new might end up in a failure. This inconvenience paired with personal habits let the potential users stick to the same 5-7 dishes they know.

\subsection*{Gains}
Our persona would like to live a healthier lifestyle but it shouldn’t feel like a sacrifice. Suggested alternatives and changes should feel equally good and not inferior. It wishes for convenience instead of the usual handiwork, commonly going into keeping track of one's diet. Additionally, there is a need for clear, trustworthy guidelines rooted in scientific research.
