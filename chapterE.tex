\chapter{Limitations}

We divided the limitations in obstacles we faced during the development of Foodo and in limitations of the application in its current state.

\section{Obstacles}

\subsection{Pre-existing standards}
One of the main limitations we have been faced with, was the lack of a database with substitutes, one could build upon. Even if we would have decided to implement a learning algorithm computing substitutes, we would have still be missing a test set to verify the results. Information about possible replacements is scattered around the internet in cooking blogs and forums. These are mostly directed at vegans, vegetarians and people suffering from allergies. Additionally, these recommendations are often linked to a certain recipe, making it harder to generalize possible relationships.
The same applies to algorithms or even guidelines and rules of thumb.

\subsection{Context}
When it comes to assessing how well an ingredient performs as a substitute, there are multiple properties that have to be considered. Namely the consistency, texture, taste, look and in our case especially the nutritional value. The impact the ingredient has on the consistency and texture of the dish has to be taken into account, otherwise a stew might get turned into a soup. The toughest nut to crack is the adjustment of the list of possible substitutes to the context of a dish. With context being the symbiosis between taste and texture and therefore something hard to assess in digital terms. Applesauce as an egg replacement can work in the context of a cake, but not in an omelette. Furthermore, an egg can have multiple purposes or functions within a dish. It can work as a binding or loosening agent and add taste, color and moisture. A fully fletched substitution-algorithm would have to infer the purpose of the ingredient as well as the context of the dish based on the provided recipe. Additionally, substitutes which aren’t fit for the preparation conditions (e.g. heat) have to be filtered beforehand. 


\section{Current limits of the programm}
Every limit demands actions from the user, to compensate for the lack of automation. The more the app evolves, the less input is necessary.

\subsection{How good is the substitute}
As mentioned in 5.1.2, there are multiple properties influencing the suitability of a substitute. At the moment we are only considering the category of a food item, as well as its nutritional value. If it fits in regards of taste and texture has to be decided by the user.

\subsection{Preparation steps}
Currently we don't provide preparation steps to guide a user. Because a major issues (besides gathering the data) is to assess how the preparation changes through a substitute. The time a dish needs to stay in the oven might differ, or even the way the substitute is prepared (sliced, smashed, peeled). This can often be easily inferred by the user, but it's hard for a program to do so.

\subsection{How much should be substituted}
Currently we provide information based on the assumption that an ingredient is completely substituted. But there are cases, in which one would only like to substitute a certain percentage or use a mixture out of different alternatives. Further, some substitutes only work as a 'thinner'. Next to this it always has to be considered, how much has been substituted already. Applesauce for example, can be used as an replacement for eggs, butter or as a natural sweetener. But it shouldn’t substitute all of them. 

\subsection{Cross-substitute effects}
The applesauce-example from above illustrates an additional detail, that should be considered in the future. Substitutes provide other properties than their respective counterparts. If we use applesauce as an egg replacement, the whole pastry is getting sweeter, which in return should lead to a decrease in the amount of sugar needed.
The effect that one substitute choice has on the amount of other ingredients used, is mainly relevant for sugar, salt and fats/oil. 

\section{Outlook}

\subsection{More of current features}
The easiest step to add more value to the current application is by extending existing features.

\subsubsection{New nutritional goals}
New goals can broaden the target audience. One could open up to the medical market by introducing a strict reduction of sugar for diabetics or lowering salt and cholesterol for people suffering from heart disease. The implementation would be fairly simply. The algorithm could filter possible substitutes by their content of salt instead of their NutriScore.

\subsubsection{Add ingredients and recipes}
Foodo should know about as much ingredients as possible, to consider them in it’s algorithm. This directly improves the proposed substitutes, because there are more alternatives to choose from. New recipes on the other hand are essential to attract and keep users. We claimed to provide users with their favorite recipes, to spare them from creating them manually. By delivering on this promise, we can embrace Foodos convenience.

\subsection{Community}
The task of adding recipes and ingredients to the database could be transferred to the user.
With a rating system popular recipes could rise to the top and bad ones would be sorted out. This system could also be applied to the substitutes, where users like or dislike existing ones and can propose new ones. The data which substitute a user picks and if he liked it or not is already collected and just has to be aggregated and turned into a popularity ranking.
With features like comment sections, users could exchange more tips regarding certain substitutes (how good is it, how to prepare it, how much of the original ingredient should be replaced with it). That way, a first collection of substitutes could emerge, supporting further research in this field. Moreover, the focus on the community would get users more involved and keep them attached to foodo.

\subsection{Recommendations}
Once there is a stable user base, regularly interacting with foodo, the arising data can be used for analytical purposes. Besides optimizing substitution proposals, new recipes could be recommended to users, based on what they have already tried and liked. Adapting to a users preferences would lead to a more personalized experience.

\subsection{Complements}
An easier approach to recipe improvement is to use complementary ingredients instead of substitutes. Complements only have to fit well to the existing dish, but don't have to replace an ingredient in multiple regards (binding/loosening properties, moisture, taste, color). 
For example peppers fit well to a quiche, stir-fried vegetables or ratatouille but are in no regard a substitute for one of the ingredients in these dishes. The same applies to adding a handful of peas to pork with noodles. An addition that can’t be considered a replacement but is still an improvement, at least indirectly. Because the peas increase the total weight and therefore the relative composition of the dish. This can be directly seen in the nutritional information of a single serving (example/ figure below). 

Of course this doesn’t work out if still the same amount of pork and pasta are eaten and the peas just come on top. But in theory one would reduce either the pork, the noodles or both, to maintain the same size for a single serving.

\subsection{Take less}
The aforementioned idea of complements builds upon reducing the amount of (unhealthy) ingredients using supplements instead. But there are also use cases in which neither substitutes nor supplements are needed to improve a recipe. Especially in the context of baking there are often huge amounts of sugar used, which can be reduced without having any noticeable impact on the pastries. The dispensable amount of the unhealthy ingredient could either be provided and verified by the community or automatically inferred by comparing other recipes of the same dish.

\subsection{Business model}
In the future it’s also worth thinking of new income streams, besides the current premium membership on a voluntary basis. One source could be advertisements, like banner ads or product placements. Kitchen gadgets or ingredients by a certain brand could be mentioned within recipes or their preparation steps. 
Partnerships with food retailers would provide a stable income and additional input for the ingredient database. Edeka, Rewe and their competitors are already storing nutritional facts, geographical information (where the product comes from) as well as the current price and could grant us access to them. With these information optimizing a recipe in terms of overall costs or eco-friendliness (prefer regional or seasonal products) could become a new feature. 
Further, this would make it easier to implement a shopping list, directly linked to a partners delivery service.