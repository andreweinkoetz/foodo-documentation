\chapter{Description of Idea}
\begin{center}
\textit{Foodo - Your Loyal Companion on Your Way to a Healthier Lifestyle}
\end{center}

Foodo offers the user a collection of standard recipes. A subset of these standard recipes represents the user's recipe stock, namely the dishes the user is used to cook regularly. The main goal of Foodo is to iteratively extend and to improve the healthiness of the user's recipe stock in a personal way. This is achieved by enabling the user to substitute the recipes' ingredients with suggested healthy substitutes leading to healthier variations of the user's recipes. Moreover, Foodo's collection of standard recipes offers the user the opportunity to extend his recipe stock by trying out new recipes. 

As nutrition is very individual, Foodo is a personal companion and adjusts its suggestions to the characteristics of the user. The user can set personal preferences. These include his lifestyle, e.g. Vegetarian, Vegan or Low Carb, his nutrition goal, e.g. eating healthy, reducing weight or gaining muscles, his dislikes and his allergies, e.g. Gluten, Lactose or Fructose. After setting these personal preferences, Foodo takes them into consideration and finds the best personal suggestions for the user's recipe stock. If the user accepts a suggested substitution, the recipe gets updated and becomes part of his personal recipe stock. Moreover, the user can give feedback on the substitution after he has cooked the recipe which has been modified with the substitution. Foodo will remember his feedback and incorporate it in its future suggestions. Like this, the suggestions get even more personalized over time.

Foodo offers different channels to interact with the user. These are a webapp and an Alexa skill. The webapp offers detailed information on the available recipes, the user's recipe stock, his preferences, the recipes, their nutrition values and statistics on achieved improvements by using Foodo. This channel targets the sophisticated user and is additionally used for fulfilling administrative tasks like inserting the user's preferences. The Alexa skill focuses on the substitution process and on supporting the cooking activity. Therefore, it does not offer statistics or too detailed information but it guides the user through the core functionalities in a supportive way. This channel targets the casual user and also allows the usage of Foodo while cooking as it only requires voice for the interaction and thus does not interfere with the cooking activity.


\section{Pain Relievers}
Foodo relieves several user pains. Firstly, Foodo relies on a user-centric approach which is compared to a recipe-centric approach more personal and less frustrating for the user. One aspect of the user-centric approach is that the required user input is reduced to a minimum. Another aspect is the Alexa integration which allows an engaging user interaction through voice commands which reduces the time overhead. Furthermore, Foodo lowers the complexity of finding the right nutrition for the user. Through specifying a nutrition goal, the user does not need to think about how to achieve this goal but rather just think about what goal the user wants to achieve. Foodo also reduces the complexity of finding fitting substitutions by suggesting the best suitable substitutes to the user. Apart from that, the user does not have to search for new recipes anymore if he wants to extend his recipe stock because Foodo suggests the user new recipes from its standard recipe collection. This saves the user a lot of time. Moreover, Foodo enables guided cooking experiments and thereby takes the user's fear of bad results which can occur when the user varies the ingredients of recipes without guidance.


\section{Gain Creators}
Foodo offers several gain creators to the user. The most important gain creator for the user is that its recipe stock overall gets continuously healthier and gets extended over time. The incremental improvements of the amount and the variety of the user's recipe stock establish an awareness for a healthy nutrition for the user. By cooking healthy and getting suggestions for substitutions tailored to the user's preferences, the user also gains knowledge about healthy nutrition. Moreover, Foodo improves the user's cooking experience, as the Alexa integration supports the cooking activity and offers a more convenient interaction than having to use a smartphone which might be difficult while handling ingredients and cooking.


